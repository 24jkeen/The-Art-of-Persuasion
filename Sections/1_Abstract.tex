\begin{abstract}
\thispagestyle{plain}
It is integral to the success of a multi-agent system that the agents cooperate in order to achieve the best possible outcome for the group. Agents must share their beliefs to reach a unanimous agreement upon the best policy the population should follow. Current methods in published literature require an agent to broadcast its complete set of beliefs. This is a loss of both precision and privacy. To address this, the first four models in this paper structure the communication created by an agent. This incorporates aspects of persuasion by only sharing the most relevant information, consequently improving the explainability of a collective decision. 

The subsequent four models describe the different reactions an agent can have upon receiving new information. It is shown both mathematically and in simulation that the Bottom Up and Top Down models converge to a unanimous agreement in a single state of the world. This is only achieved with Passive listeners. When the listeners are capable of disregarding information they deem unlikely, the population polarises. Furthermore, the Optimised model that makes any argument it must in order to persuade a listening agent is shown to be ineffective at regularly producing a consensus in a single state. These models demonstrate that it is possible to incorporate aspects of persuasion to improve communication between agents. 

\end{abstract}