\begin{abstract}

It is integral to the success of a multi-agent population that each agent works toward the collective good of the group. This requires each agent to agree on the best possible outcome. To do this, agents share and combine their beliefs in order to form a consensus. However, the methods proposed in literature often require an agent to broadcast its complete set of beliefs, losing both privacy and precision. In this paper, four models are proposed that define a persuasive structure to the communication an agent makes. A further four models are proposed that detail an agent's possible reaction to such a communication. It is shown that the models that construct an argument based on what the agent believes regularly converge, while the ethically questionable model in which an agent constructs any argument it must so as to persuade the listener to agree with the agent is less practical. However, when the listener is able to reject an argument, the population polarises, forming distinct, isolated groups with shared, extreme beliefs. These models are simplistic, though demonstrate that aspects of persuasion can be included in communication in multi-agent systems.


\end{abstract}